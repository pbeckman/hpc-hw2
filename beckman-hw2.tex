\documentclass{article}
\usepackage{parskip}
\usepackage{amsmath, amsthm, amssymb, amsfonts, mathtools, xfrac, dsfont}
\usepackage[margin=0.8in]{geometry}
\usepackage{hyperref}
\usepackage{bm}
\usepackage{xcolor}
\usepackage{float}
\usepackage[justification=centering,labelfont=bf]{caption}

\newcommand{\pr}[1]{\left(#1\right)}
\newcommand{\br}[1]{\left[#1\right]}
\newcommand{\norm}[1]{\left\lVert#1\right\rVert}
\newcommand{\ip}[1]{\left\langle#1\right\rangle}
\renewcommand{\vec}[1]{\left\langle#1\right\rangle}
\newcommand{\ind}[1]{\mathds{1}_{#1}}
\newcommand{\abs}[1]{\left|#1\right|}
\newcommand{\mat}[1]{\begin{bmatrix}#1\end{bmatrix}}
\newcommand{\der}[2]{\frac{d #1}{d #2}} \newcommand{\mder}[2]{\frac{D #1}{D #2}}
\newcommand{\pder}[2]{\frac{\partial #1}{\partial #2}}
\renewcommand{\arraystretch}{1.3}
\newcommand{\R}{\mathbb{R}} \newcommand{\Q}{\mathbb{Q}}
\newcommand{\C}{\mathbb{C}} \newcommand{\N}{\mathbb{N}}
\newcommand{\Z}{\mathbb{Z}} \newcommand{\E}{\mathbb{E}}
\renewcommand{\P}{\mathbb{P}} \renewcommand{\S}{\mathbb{S}}
\renewcommand{\O}{\mathcal{O}} \newcommand{\F}{\mathcal{F}}
\renewcommand{\epsilon}{\varepsilon}
\DeclareMathOperator{\Cov}{Cov} \DeclareMathOperator{\Var}{Var} \let\Re\relax
\DeclareMathOperator{\Re}{Re} \let\Im\relax \DeclareMathOperator{\Im}{Im}
\DeclareMathOperator{\diag}{diag} \DeclareMathOperator{\tr}{tr}
\DeclarePairedDelimiter\floor{\lfloor}{\rfloor}
\DeclarePairedDelimiter\ceil{\lceil}{\rceil}
\DeclareMathOperator*{\argmax}{arg\,max}
\DeclareMathOperator*{\argmin}{arg\,min}

\setlength\parindent{0pt}

\title{High Performance Computing: Homework 2}
\author{Paul Beckman}
\date{}

\begin{document}

\maketitle

\section{Finding memory bugs}
For \texttt{val\_test01}, we make the following changes
\begin{itemize}
  \item line 80: change \texttt{<=} to \texttt{<} to avoid indexing out of
  bounds 
  \item line 86: change \texttt{delete []} to \texttt{free} to match original
  malloc
\end{itemize}

For \texttt{val\_test02}, we add the initialization block
\begin{verbatim}
  for ( i = 6; i < 10; i++ )
  {
    x[i] = 0;
  }
\end{verbatim}
to avoid copying and printing uninitialized variables.

\section{Optimizing matrix-matrix multiplication}
\subsection{Loop ordering}
The given loop ordering gives the following timings on a Intel(R) Xeon(R) CPU
@2.53GHz processor (\texttt{crackle1})
\begin{verbatim}
 Dimension       Time    Gflop/s       GB/s        Error
        16   0.732706   2.729614  43.673817 0.000000e+00
       208   0.772579   2.609128  41.746043 0.000000e+00
       400   0.769613   2.661077  42.577230 0.000000e+00
       592   0.783120   2.649334  42.389349 0.000000e+00
       784   1.083208   2.669241  42.707854 0.000000e+00
       976   1.373894   2.706800  43.308797 0.000000e+00
      1168   1.241814   2.566268  41.060295 0.000000e+00
      1360   2.242826   2.243112  35.889800 0.000000e+00
      1552   3.530674   2.117618  33.881884 0.000000e+00
      1744   5.011761   2.116796  33.868742 0.000000e+00
      1936   6.864092   2.114282  33.828515 0.000000e+00
\end{verbatim}
The other ordering where the inner loop is over columns performs similarly. In
contrast, other loop orderings give slower timings. For example, if we exchange
the \texttt{i} and \texttt{p} variables so that the inner loop is over the
shared dimension \texttt{k}, we obtain
\begin{verbatim}
 Dimension       Time    Gflop/s       GB/s        Error
        16   1.318957   1.516352  24.261640 0.000000e+00
       208   1.597254   1.262014  20.192218 0.000000e+00
       400   2.103392   0.973665  15.578647 0.000000e+00
       592   1.986053   1.044658  16.714533 0.000000e+00
       784   2.674992   1.080879  17.294057 0.000000e+00
       976   3.449245   1.078165  17.250648 0.000000e+00
      1168   3.529046   0.903028  14.448448 0.000000e+00
      1360   6.411109   0.784718  12.555486 0.000000e+00
      1552  10.383032   0.720080  11.521286 0.000000e+00
      1744  15.729447   0.674460  10.791354 0.000000e+00
      1936  21.918752   0.662110  10.593762 0.000000e+00
\end{verbatim}
This can be explained by the column major ordering of the matrix storage, as
having the column variable in the inner loop allows one or more columns to be
cashed during computation, reducing memory access requirements.

\subsection{Blocking}
After blocking (with \texttt{BLOCK\_SIZE} 16) we see immediate speedups and
increased bandwidth, as more arithmetic is done using cached matrix entries
\begin{verbatim}
 Dimension       Time    Gflop/s       GB/s        Error
        16   0.000004   2.278087  36.449388 0.000000e+00
       208   0.006020   2.989627  47.834033 0.000000e+00
       400   0.042895   2.984037  47.744597 0.000000e+00
       592   0.139096   2.983178  47.730855 0.000000e+00
       784   0.328163   2.936894  46.990312 0.000000e+00
       976   0.629938   2.951764  47.228223 0.000000e+00
      1168   1.133763   2.810842  44.973469 0.000000e+00
      1360   1.950271   2.579597  41.273548 0.000000e+00
      1552   2.978048   2.510576  40.169218 0.000000e+00
      1744   4.221113   2.513289  40.212627 0.000000e+00
      1936   5.799208   2.502519  40.040300 0.000000e+00
\end{verbatim}
In this regime, smaller block sizes appear to be best for speed. Using
\texttt{BLOCK\_SIZE} 4 gives
\begin{verbatim}
 Dimension       Time    Gflop/s       GB/s        Error
         4   0.000001   0.139891   2.238251 0.000000e+00
       204   0.004475   3.794272  60.708348 0.000000e+00
       404   0.034631   3.808085  60.929359 0.000000e+00
       604   0.117891   3.738184  59.810939 0.000000e+00
       804   0.281168   3.696848  59.149565 0.000000e+00
      1004   0.548550   3.689902  59.038437 0.000000e+00
      1204   0.951046   3.670355  58.725673 0.000000e+00
      1404   1.560903   3.546135  56.738164 0.000000e+00
      1604   2.343953   3.521229  56.339658 0.000000e+00
      1804   3.358989   3.495675  55.930796 0.000000e+00
\end{verbatim}
which shows improvement over \texttt{BLOCK\_SIZE} 16. If we increase the block
size above 16, we see even slower results. This is a bit surprising, as I would
expect a larger block size to ``just barely fit" in the cache and thus give
optimal performance.

\subsection{Parallelism}
Following the discussion in lecture, I tried reordering the loops in each block
so that the shared dimension \texttt{k} is the inner loop and using
\texttt{collapse(2)} on the outer two loops as they are perfectly nested.
However, this appears to lead to serious slowdowns. 

Alternatively, taking the most naive approach and simply slapping a
\texttt{parallel for} in front of the first outer loop gives decent speedups.
For example, with a bit larger \texttt{BLOCK\_SIZE} 16 and 16 threads, we obtain 
\begin{verbatim}
 Dimension       Time    Gflop/s       GB/s        Error
        16   0.000158   0.051758   0.828133 0.000000e+00
       208   0.001721  10.457425 167.318795 0.000000e+00
       400   0.010832  11.817296 189.076738 0.000000e+00
       592   0.034267  12.109437 193.750996 0.000000e+00
       784   0.078501  12.277328 196.437248 0.000000e+00
       976   0.145595  12.771261 204.340171 0.000000e+00
      1168   0.261762  12.174537 194.792598 0.000000e+00
      1360   0.454422  11.071019 177.136298 0.000000e+00
      1552   0.739173  10.114843 161.837495 0.000000e+00
      1744   1.074908   9.869567 157.913073 0.000000e+00
      1936   1.463838   9.914097 158.625553 0.000000e+00
\end{verbatim}
which is notably faster than the serial blocked implementation, but far from
linear strong scaling.

\subsection*{3}

\subsection*{4}

\end{document}
